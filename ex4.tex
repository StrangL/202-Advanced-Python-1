\documentclass[a5paper,8pt]{article}
\usepackage[9pt]{extsizes}
\usepackage[utf8]{inputenc}
\usepackage[T2A]{fontenc}
\usepackage{geometry}
\geometry{a5paper,right=20mm, left=20mm, tmargin=20mm, bmargin=20mm, headsep=4mm}
\usepackage{float}
\usepackage{amsmath}
\usepackage{babel}

\setcounter{page}{19}
\usepackage{fancyhdr}
\fancyhead[R]{\thepage}
\fancyhead[L]{}
\fancyhead[C]{\textit{\textsection 1.1 Релятивистcкая волновая механика}}
\fancyfoot[L]{2*}
\fancyfoot[C]{}

\begin{document}
\sloppy
\pagestyle{fancy}
\noindent
опубликовать свое релятивистское волновое уравнение, оно уже было независимо переоткрыто Оскаром Клейном [7] и Вальтером Гордoном  [8]. По этой причине релятивистский вариант называется "уравнением Клейна-Гордона".

Шредингер вывел свое релятивистское волновое уравнение, заметив, что гамильтониан $H$ и импульс $\mathbf{p}$ ``электрона Лоренца'' с массой $m$ и зарядом $e$, находящегося во внешнем векторном потенциале $\mathbf{A}$ и кулоновском потенциале $\phi$, связаны следующим соотношением \footnote{)Это соотношение лоренц-инвариантно, поскольку величины $\mathbf{A}$ и $\phi$ при преобразованиях Лоренца изменяются точно так же, как $c \mathbf{p}$ и $E$ соответственно. Гамильтониан $H$ и импульс $\mathbf{p}$ Шредингер представлял в виде частных производных действия, однако это неважно для нашего рассмотрения.}$^)$:

\begin{equation}
\label{formula1}
0 = (H + e\phi )^2 - c^2 (\mathbf{p} - e\mathbf{A}/c)^2 - m^2 c^4 .
\end{equation}
\noindent
Соотношения де Бройля для \textit{cвободной} частицы, представленной плоской волной $\exp\{2 \pi i(\boldsymbol{\kappa}\cdot\textbf{x}-v t)\} $, можно получить, если произвести отождествление
\begin{equation}
\label{formula2}
\textbf{p} = h\textbf{k}\rightarrow -i\hbar\nabla,\indent E=h\nu \rightarrow i\hbar \frac{\partial}{\partial t},
\end{equation}
\noindent
где $\hbar$ -- удобное обозначение (введенное Дираком) для $h/2\pi$. Исходя из чисто формальной аналогии Шредингер предположил, что электрон во внешних полях \textbf{A}, $\phi$ должен описываться волновой функцией $\psi(\mathbf{x}, t)$, удовлетворяющей уравнению, получаемому при помощи той же самой замены в (\ref{formula1}):
\begin{equation}
\label{formula3}
0 = \left[\left(i\hbar\frac{\partial}{\partial t} + e\phi\right)^2 - c^2\left(-i\hbar\nabla +\frac{e\mathbf{A}}{c}\right)^2 - m^2 c^4\right]\psi(\mathbf{x}, t) .
\end{equation}
В частности, для стационарных состояний в атоме водорода справедливы равенства $\mathbf{A} = 0$ и $\phi = e/(4\pi r)$. Кроме того, в этом случае $\psi$ зависит от времени $t$ экспоненциально: $\exp(-iEt/\hbar)$. Поэтому (\ref{formula3}) сводится к уравнению
\begin{equation}
\label{formula4}
0 = \left[\left(E +\frac{e^2}{4\pi r}\right)^2 - c^2\hbar^2\nabla^2 - m^2 c^4\right]\psi(\mathbf{x}) .
\end{equation}
Решения уравнения (\ref{formula4}) с наложенными на них разумными граничными условиями, определяют уровни энергии [9]
\begin{equation}
\label{formula5}
E = m c^2 \left[1 - \frac{\alpha}{2 n^2} - \frac{\alpha^2}{2 n^4} \left(\frac{n}{l + 1/2} - \frac{3}{4}\right) + \ldots\right],
\end{equation}
где $\alpha\equiv e^2 /(4\pi\hbar c)$ -- ``постоянная тонкой структуры'', численное значение которой составляет приблизительно $1/137$, $n$ -- положительнео целое число, а $l$ -- орбитальный угловой момент в единицах $\hbar$, принимающий целочисленные значения в интервале $0\le l\le n - 1$. Наличие
\end{document}